\documentclass{article}
\usepackage{amssymb,amsmath,color,thumbpdf}
\definecolor{rltred}{rgb}{0.75,0,0}
\definecolor{rltgreen}{rgb}{0,0.5,0}
\definecolor{rltblue}{rgb}{0,0,0.75}
\usepackage[pdftex,
        colorlinks=true,
        urlcolor=rltblue,       % \href{...}{...} external (URL)
        filecolor=rltgreen,     % \href{...} local file
        linkcolor=rltred,       % \ref{...} and \pageref{...}
        pdftitle={Riddles},
        pdfauthor={Mark Veltzer},
        pdfsubject={A collection of riddles},
        pdfkeywords={riddles, Mark Veltzer, collection},
        pdfproducer={pdfLaTeX},
        pagebackref,
        pdfpagemode=UseNone,
        bookmarksopen=true]{hyperref}

\title{Riddles}
\author{Mark Veltzer}
\date{\today}

\begin{document}

\maketitle

\section{Rational points on a circle}

\subsection{Question}

Could there be a circle on the plane where only one point has rational co-ordinates? A point has rational co-ordinates if $x,y\in\mathbb{Q}$. What about a sphere? Are there circles with more than one rational point? Are there circles with infinite rational points? Can you find a circle with exactly $n$ rational points on it for each $n\in\mathbb{N}$?

\subsection{Solution}

Yes, there could. Consider the circle: $(x-r)^2+y^2=r^2$ for some irrational $r$.
$(0,0)$ is a point on this circle and is rational. But from the equation it follows
that: $x^2-2xr+r^2+y^2=r^2$ or $x^2+y^2=2xr$ or $r=(x^2+y^2)/2x$. If there
was a rational solution to this equation where $x\ne0$ then it would follow that
$r$ is rational since it is the result of multiplication, division, addition and subtration
of rational numbers and this is a contradiction. This means this equation can only
have rational solutions for $x=0$. If $x=0$ then $r^2+y^2=r^2$ and so $y=0$ and so
$(0,0)$ is the only such solution.

Same solution applies to a sphere: $(x-r)^2+y^2+z^2=r^2$ for some irrational $r$.
In this case: $r=(x^2+y^2+z^2)/2x$. Which means that for $x\ne0$ any rational soultion would imply a rational $r$.
$(0,0,0)$ is therefor the only rational solution.

Yes, there are circles with more than one rational point. Take $x^2+y^2=2$ which is the circle whose
center is at $(0,0)$ and whose radius is $\sqrt{2}$. The points $(1,1),(-1,1),(1,-1),(-1,-1)$ are four rational
points which are on this circle.

Yes, there are circles with infinite rational points.

\section{Always lead in election (combinatorics)}

\subsection{Question}

An election was voted a perfect tie (even number of electors) and was decided by way of putting
all the votes into a hat, mixing them uniformly and counting them one by one. What is the chance
that one of the candidates was leading during the entire counting process?

\subsection{Solution}
The question is really how many graphs that go either one step up or one step down (lets call those
binary graphs) do not go down below zero out of the set of all graphs that go from zero to zero.
The solution for n seems to be 1/((n/2)+1) (this was derived numerically).

\label{end}\end{document}
